\documentclass[a4paper,oneside,12pt]{report}

%%%%%%%%%%%%%%%%%%% LOAD PACKAGES %%%%%%%%%%%%%%%%%%%%%
%Get support for äöüßÄÖÜ
\usepackage[utf8]{inputenc}
%Get support for word separation
\usepackage[T1]{fontenc}
%Direkte Verwendung von Umlauten
\usepackage[ansinew]{luainputenc}
%Get Support for US english and german
\usepackage[english,ngerman]{babel}
%verbesserter Randausgleich
\usepackage{microtype}
%zulässiger Wortzwischenraum erhöhen für noch besseren Randausgleich und Blocksatzbildung
\setlength\emergencystretch{1em}
%Einrücken nach Absatz für gesamtes Dokument abstellen
\setlength{\parindent}{0pt} 
%Bildbeschreibung zentrieren
\usepackage[center]{caption}
%Support for Graphics .png .jpg .pdf 
\usepackage{graphicx}
%Zusätzliche Option[H] bei Bildern - Hier und und sonst nirgends!
\usepackage{float}
%Adjust the page margins asymmetrical, Need to be before subfig
\usepackage[left=3.5cm, right=2.5cm, top=3cm, bottom=3cm, a4paper, portrait]{geometry}
%Kopf- und Fußzeilen
\usepackage{fancyhdr}
%für Fußnote (linksbündig)
\usepackage[flushmargin,bottom]{footmisc}
%Erweiterung des Mathematikmoduses
\usepackage{amsmath}
%Use Palatino Linotype 
\usepackage{mathpazo}
%Don´t show "Kapitel 1" at \chapter
\usepackage{titlesec} 
\titleformat{\chapter}{\bfseries\Huge}{\thechapter\quad}{0em}{}
%Abstand Kapitelüberschiften zum Kopfrand
\titlespacing{\chapter}{0pt}{*-2}{*1.5}
\usepackage{setspace}
\AtBeginDocument{\renewcommand{\chaptername}{}}
%Tabellenspalten/zeilen einfärben
\usepackage{colortbl}
%variable Tabellenbreite
\usepackage{tabularx}
%Multirow in Tabelle
\usepackage{multirow}
%Tabellenüberschrift links oben!
\usepackage[singlelinecheck=false]{caption}
%Programmcode einfügen
\usepackage{listings}
%Programmcode Farben ändern
\usepackage{xcolor}
%Hyperlink
\usepackage[colorlinks=true,linkcolor=black, citecolor=black, urlcolor=black]{hyperref}
%Ordnerstruktur erstellen
\usepackage{dirtree}
\usepackage{subfigure}
%Abkuerzungen
\usepackage{acronym}
%Highlighting
\usepackage{color,soul}
%Unterstützung für textumflossenes Bild 
\usepackage{wrapfig}


%//%%%%%%%%%%%%%%%%%% END LOAD PACKAGES %%%%%%%%%%%%%%%%%%%%
%//%%%%%%%%%%%%%%%%%% SETTINGS %%%%%%%%%%%%%%%%%%%%%%%%%
%Adjust headings and footers --> \usepackage{fancyhdr}
%Give the headings some space
\setlength{\headheight}{20pt}
%This is valid for all pages exept chapters
\pagestyle{fancy}

\fancyhf{} % clear all headers and footers
%Ausrichtung Kopf- /Fußzeile
\lhead[]{\fancyplain{}{\leftmark}}
\cfoot[]{\thepage}
\renewcommand{\headrulewidth}{0.4pt}
%For \chapters \maketitle
\fancypagestyle{plain}{%
	\fancyhf{} % clear all header and footer fields
	\cfoot[]{\thepage}%
	\renewcommand{\headrulewidth}{0pt}
	\renewcommand{\footrulewidth}{0pt}
}
%Deactivate command --> small letter
\let\MakeUppercase\relax
%Tiefe von Inhaltsverzeichnis 
\setcounter{secnumdepth}{3}
\setcounter{tocdepth}{3}
%Programmcodeeinstellungen
\lstset{
	backgroundcolor=\color{black!5},
	tabsize=4,
	language=C++,
	captionpos=b,
	tabsize=3,
	frame=lines,
	numbers=left,
	numberstyle=\tiny,
	numbersep=5pt,
	breaklines=true,
	showstringspaces=false,
	basicstyle=\tiny,
	%identifierstyle=\color{magenta},
	keywordstyle=\color[rgb]{0.8,0.4,0},
	commentstyle=\color[rgb]{0.9,0.3,0.9,},
	stringstyle=\color{green},
	emphstyle=\color{blue},
}
%Programmcodeeinstellungen, style Arduino
	\lstdefinestyle{Arduino}{%
		%style=numbers,
		keywords={soup},%                 define keywords
		%morestring=[s]{<}{>},%			  define <> as strings
		morecomment=[l]{\#},%             treat // as comments
		morecomment=[s]{/*}{*/},%         define /* ... */ comments
		emph={void, for, else, if, in, HIGH, OUTPUT, LOW, int, uint8_t, self, def, True, False, print, from, import, as, class, try, except, not, or, return, global},%        keywords to emphasize	
}


%Bennenung von Caption bei Programmcode 
	\renewcommand{\lstlistingname}{Programmcode}
	\renewcommand{\lstlistlistingname}{\lstlistingname}
	%Bennenung von Bilderquelle als "Quelle:"
	\newcommand*{\quelle}{%
		\footnotesize Quelle:
	}

%//%%%%%%%%%%%%%%%%%% END SETTINGS %%%%%%%%%%%%%%%%%%%%%%%%%

%\\%%%%%%%%%%%%%%%%%% DOCUMENT %%%%%%%%%%%%%%%%%%%%%%%%%


\begin{document}
	%%%%Bearbeitet
%Deckblatt
\thispagestyle{empty}
\begin{center}
	\includegraphics[scale=0.5]{Bilder/HM1_logo.png}


	
	\vspace{1.5cm}
	\huge{Programmcode}\\\vspace{1.5cm}
	\large{im Zusammenhang mit der Bachelorarbeit}\\\vspace{0.5cm}
	\large{an der}\\\vspace{0.5cm}
	
	\huge{Hochschule München}\\
	\normalsize {für angewandte Wissenschaften}\\\vspace{0.5cm}
	\large{mit dem Titel}\\\vspace{1cm}
    \begin{center}
		\Huge{\textbf{Aufbau eines automatischen Testers für ein automotive Messtechnik Rack}}\\[2.0cm]
    \end{center}
    
	
	
	\large{Fakultät für Wirtschaftsingenieurwesen}\\
	\normalsize {Studiengang Wirtschaftsingenieurwesen, Schwerpunkt industrielle Technik}\\\vspace{2.1cm}
	
\end{center}

	
\newpage
\thispagestyle{empty}
\begin{flushleft}	
	\begin{tabular}[H]{ll}
		
		Ersteller: & \large{Johannes Knippel, Anja Wolf,}\\[0.2cm]
		 			& \large{Johanna Sickendiek, Skanny}\\[0,2cm]
		1. Prüfer:		  & \large{blablablabla}\\[0.2cm]
		2. Prüfer:		  & \large{loablebli}\\[.7cm]
		
		
		Betreuer an der Hochschule München: 	& \large{Prof. Dr. Rainer Schmidt}\\[0.5cm]
		Ausgabedatum:					& \large{19.03.2018}\\[0.5cm]
		Abgabedatum:					& \large{22.07.2018}\\[3.5cm]
	
		
		%FeldLinks für OrtDatum1	
		%\hspace*{\fill}
	
		\parbox{7cm}{\hrule\medskip Ort, Datum\\ [0.3cm]}	&	\parbox{7cm}{\hrule\medskip Unterschrift des 1. Prüfers\\ [0.2cm]Blaaaaaaaaaaaabla}\\[1.5cm]
		\parbox{7cm}{\hrule\medskip Ort, Datum\\ [0.31cm]}	&	\parbox{7cm}{\hrule\medskip Unterschrift des 2. Prüfers\\ [0.2cm]Blaaaaaaaaaaaabla}\\[1.5cm]
		\parbox{7cm}{\hrule\medskip Ort, Datum\\ [0.33cm]}	&	\parbox{7cm}{\hrule\medskip Unterschrift des Verfassers\\ [0.2cm]Johannes Knippel}\\[1.5cm]
		\parbox{7cm}{\hrule\medskip Ort, Datum\\ [0.33cm]}	&	\parbox{7cm}{\hrule\medskip Unterschrift des Verfassers\\ [0.2cm]Anja Wolf}\\[1.5cm]
		\parbox{7cm}{\hrule\medskip Ort, Datum\\ [0.33cm]}	&	\parbox{7cm}{\hrule\medskip Unterschrift des Verfassers\\ [0.2cm]Johanna Sickendiek}\\[1.5cm]
		\parbox{7cm}{\hrule\medskip Ort, Datum\\ [0.33cm]}	&	\parbox{7cm}{\hrule\medskip Unterschrift des Verfassers\\ [0.2cm]Skanny}\\
	\end{tabular}
\end{flushleft}

	
	
	

	\pagenumbering{Roman}


%												 Inhaltsverzeichnis
	\newpage
	\tableofcontents
	
	
	
	
	
%											    Kapitel 1 - Einleitung
	%\newpage
	\pagenumbering{arabic}
	\setstretch{1.5}
	\chapter{Blablabla}\label{bla}
		
		Programmiersprache: Python
		hihii
				
			\begin{lstlisting} [caption={Registerprogrammierung der Expander}\label{code-i2c-1}, captionpos=b, style=Arduino]
			# -*- coding: utf-8 -*-
			
			import requests
			import os
			import unicodedata
			import validators
			import threading
			from tkinter import *
			from tkinter import messagebox
			
			from bs4 import BeautifulSoup
			from setuptools.package_index import HREF
			from tinydb import TinyDB,Query,where
			
			
			
			
			
			class Web_scraping:
			
			'''
			Created on 17.04.2018
			
			@author: anjawolf
			
			get_single_data: Data will be scraped from the Tripadvisors Website and stored in variables.
			'''
			def get_single_data(self,url):
			source_code = requests.get(url)
			plain_text = source_code.text
			soup = BeautifulSoup(plain_text, "html.parser")
			
			for self.name in soup.find('h1', {'class': 'heading_title'}):
			print("Name:" + self.name.string)
			
			for self.rating_amount in soup.find('span', {'property': 'count'}):
			print("Anzahl Bewertungen:" + self.rating_amount.string)
			
			for self.popularity in soup.find_all('span', {'class': 'header_popularity popIndexValidation'}):
			print("Popularitaet:" + self.popularity.text)
			
			for self.price_level in soup.find('span', {'class': 'header_tags rating_and_popularity'}):
			print("Preis Level:" + self.price_level.string)
			
			for self.cuisine in soup.find_all('span', {'class': 'header_links rating_and_popularity'}):
			print('Kueche:' + self.cuisine.text)
			
			for self.contact_details in soup.find('div', {'class': 'blRow'}):
			print('Kontaktdaten:' + self.contact_details.text)
			
			for self.address in soup.find('span', {'class': 'street-address'}):
			print("Strasse:" + self.address.string)
			
			for self.locality in soup.find('span', {'class': 'locality'}):
			print("PLZ:" + self.locality.string)
			
			for self.phonenumber in soup.find_all('div', {'class': 'blEntry phone'}):
			print("Telefonnummer:" + self.phonenumber.text)
			
			
			
			#get all the review containers from current page
			review_containers = soup.find_all('div', class_= 'review-container')
			reviewsPerPage = len(review_containers)
			#print (reviewsPerPage)
			
			#get the pageNumbers for all reviews per restaurant
			pageNumbers = soup.find_all('a', class_= 'pageNum')
			pages = len(pageNumbers)
			
			#Extract data from single containers   
			for container in review_containers:
			
			for link in container.find_all('a', href=True):
			href = "https://www.tripadvisor.de" + str(link.get('href'))
			#pass it to the function where single data is scraped
			self.get_single_review_data(href)
			
			#create a new_url for each page of the reviews
			for index in range (10,pages*10, 10):
			
			data = url.split("Reviews-")
			new_url = data[0]+"Reviews-or"+str(index)+'-'+data[1]
			#print(new_url)
			#pass the new_url to the loop_trough_review_pages function
			self.loop_through_review_pages(new_url)
			
			
			
			
			
			#loop through the restaurant reviews
			def loop_through_review_pages(self, loop_url):
			
			source_code = requests.get(loop_url)
			plain_text = source_code.text
			soup = BeautifulSoup(plain_text, "html.parser")
			
			#get all review containers but on the nuw_url
			review_containers = soup.find_all('div', class_= 'review-container')
			reviewsPerPage = len(review_containers)
			
			#search trough all reviews on the page
			for container in review_containers:
			
			for link in container.find_all('a', href=True):
			href = "https://www.tripadvisor.de" + str(link.get('href'))
			#pass it to the function where single data is scraped
			self.get_single_review_data(href)
			
			
			
			
			
			
			
			def get_single_review_data(self,review_url):
			
			source_code2 = requests.get(review_url)
			plain_text2 = source_code2.text
			soup = BeautifulSoup(plain_text2, "html.parser")
			
			#Lists to store scraped data in
			
			usernames = []
			noReviews = []
			titles = []
			content = []
			ratings = []
			
			self.titles = soup.find('div', {'id': 'PAGEHEADING'}).text
			titles.append(self.titles)
			#print('Titel:' + self.title.text)
			
			self.content = soup.find('p', {'class': 'partial_entry'}).text
			content.append(self.content)
			
			self.usernames = soup.find('span', {'class': 'expand_inline scrname'}).text
			usernames.append(self.usernames)
			
			self.noReviews = soup.find('div', {'class:': 'memberBadgingNoText'})
			noReviews.append(self.noReviews)
			
			self.rating = soup.find('span', 'alt')
			#print(self.rating)
			
			print(usernames)
			self.threadParse(self.usernames)
			#print(noReviews)
			print(titles)
			print(content)
			print(review_url)
			
			
			#print(first_link)
			#titel = soup.find('p', {'class': 'entry'})
			
			'''
			@author: JohannaSickendiek
			
			'''
			#Get the whole text of the review
			#for self.item_name in soup.find('p'):
			#    print(self.item_name.string)
			
			#self.item_name = soup.find('p')
			#print(self.item_name.string)
			
			#for self.item_name in soup.findAll('script', {'type': 'application/ld+json'}):
			#    print(self.item_name.string)
			#self.item_name = soup.findAll('script', {'type': 'application/ld+json'})
			#for dict in self.item_name:
			# print(dict["reviewBody"])
			
			
			#1. Moeglichkeit
			#for self.item_name in soup.find('p', {'class': 'partial_entry'}):
			#    print(self.item_name.string)
			#self.item_name = soup.find('p', {'class': 'partial_entry'})
			#print(self.item_name.string)
			
			# 2. Moeglichkeit
			#table = soup.findAll('div',attrs={"class":"partial_entry"})
			#for review in table:
			#    print(review.find('p').text)
			
			#Get URL from review pictures
			for link in soup.findAll('img', {'class': 'centeredImg'}):
			self.src = link.get('src')
			print("source of picture: " + self.src)
			
			
			
			
			def helloCallBack(self, string):
			self.txt.insert(END, string + '\n')
			
			def threadParse(self, string2):
			threading.Thread(target=self.helloCallBack(string2)).start()
			
			
			'''
			Created on 18.04.2018
			
			@author: JohannesKnippel
			
			TinyDB - Database. Scraped Data from get_single_data() function will be parsed to a Database saved in .json-Format.
			'''
			#TinyDB
			def parse_to_tinydb(self):
			
			cwd = os.getcwd()
			try:
			#create the database or use the existing one
			db = TinyDB('db.json')
			#create the tables inside the database
			tableReviews = db.table('REVIEWS')
			tableUsers = db.table('USERS')
			tablePictures = db.table('PICTURES')
			tableRestaurants = db.table('RESTAURANTS')
			
			except:
			print("Error opening file")    
			
			
			
			#define the data to insert into database including an auto increment ID for each Table  
			
			
			#HANDLE REVIEWS
			#check if there already exists an entry in the REVIEWSs-table with the same name and the same address (these two attributes don't change/are not variable so there is a more constant way to check for duplicates)
			#        if tableReviews.contains((where('++') == self.name.string) & (where('address') == self.address.string)): 
			#            #pop up a message box
			#            msg = "Dieses Hotel hast du bereits gesucht und ist in der Dtaenbak hinterlegt!"
			#            popup = tk.Tk()
			#            popup.wm_title("!")
			#            label = ttk.Label(popup, text=msg)
			#            label.pack(side="top", fill="x", pady=10)
			#            B1 = ttk.Button(popup, text="Okay", command = popup.destroy)
			#            B1.pack()
			#            popup.mainloop()
			# if there is no such entry, parse the data into the corresponding table of the database and define the data to insert into database including an auto increment ID for each Table       
			#        else:
			#            idReview =   
			#            dataReviews = {'fruit':'orange', 'price':25}
			#            tableReviews.insert(dataReviews)
			
			
			
			#HANDLE USERS
			#check if there already exists an entry in the USERS-table with the same name and the same address (these two attributes don't change/are not variable so there is a more constant way to check for duplicates)
			#        if tableUsers.contains((where('name') == self.name.string) & (where('address') == self.address.string)): 
			#            #pop up a message box
			#            msg = "Dieses Hotel hast du bereits gesucht und ist in der Dtaenbak hinterlegt!"
			#            popup = tk.Tk()
			#            popup.wm_title("!")
			#            label = ttk.Label(popup, text=msg)
			#            label.pack(side="top", fill="x", pady=10)
			#            B1 = ttk.Button(popup, text="Okay", command = popup.destroy)
			#            B1.pack()
			#            popup.mainloop()
			# if there is no such entry, parse the data into the corresponding table of the database and define the data to insert into database including an auto increment ID for each Table       
			#        else:
			#            idUser = 
			#            dataUsers = {'fruit':'orange', 'price':25}  
			#            tableUsers.insert(dataUsers)
			
			
			
			#HANDLE PICTURES
			#check if there already exists an entry in the PICTURES-table with the same name and the same address (these two attributes don't change/are not variable so there is a more constant way to check for duplicates)
			#        if tablePictures.contains((where('name') == self.name.string) & (where('address') == self.address.string)): 
			#            #pop up a message box
			#            msg = "Dieses Hotel hast du bereits gesucht und ist in der Dtaenbak hinterlegt!"
			#            popup = tk.Tk()
			#            popup.wm_title("!")
			#            label = ttk.Label(popup, text=msg)
			#            label.pack(side="top", fill="x", pady=10)
			#            B1 = ttk.Button(popup, text="Okay", command = popup.destroy)
			#            B1.pack()
			#            popup.mainloop()
			# if there is no such entry, parse the data into the corresponding table of the database and define the data to insert into database including an auto increment ID for each Table       
			#        else:
			#            idPicture =   
			#            dataPictures = {'fruit':'orange', 'price':25} 
			#            tableRestaurants.insert(dataRestaurants)
			
			
			
			#HANDLE RESTAURANTS
			#check if there already exists an entry in the Restaurans-table with the same name and the same address (these two attributes don't change/are not variable so there is a more constant way to check for duplicates)
			if tableRestaurants.contains((where('name') == self.name.string) & (where('address') == self.address.string)): 
			#pop up a message box
			msg = "Dieses Hotel hast du bereits gesucht und ist in der Dtaenbak hinterlegt!"
			popup = tk.Tk()
			popup.wm_title("!")
			label = ttk.Label(popup, text=msg)
			label.pack(side="top", fill="x", pady=10)
			B1 = ttk.Button(popup, text="Okay", command = popup.destroy)
			B1.pack()
			popup.mainloop()
			# if there is no such entry, parse the data into the corresponding table of the database and define the data to insert into database including an auto increment ID for each Table       
			else:
			self.popularity2 = unicodedata.normalize("NFKD", self.popularity.text)
			self.idRestaurant = 1  
			dataRestaurants = {'id':self.idRestaurant, 'name':self.name.string, 'rating_amount':self.rating_amount.string, 'popularity':self.popularity2,'price_level':self.price_level.string, 'cuisine':self.cuisine.text, 'contact_details':self.contact_details.text, 'address':self.address.string, 'locality':self.locality.string, 'phonenumber':self.phonenumber.text}
			tableRestaurants.insert(dataRestaurants) 
			
			#################################bis hier 19.04.18###############
			# Idee: 4 tabellen mit drei IDs: Hotel_ID, User_ID, Review_ID
			# Implementieren dieser IDs per auto increment funktion.
			
			
			
			
			print(tableRestaurants.all())
			ft = Query()
			suche = tableRestaurants.search(ft.name == self.name.string)
			print(suche)         
			
			
			'''
			@author: Skanny Morandi
			
			opens a GUI that validates a given url-string and starts the scraping on button click
			'''
			
			def start_GUI(self):
			
			roots = Tk()
			roots.title('Tripadvisor Scraper')
			instruction = Label(roots, text='Please provide Restaurant-Url\n')
			instruction.grid(row=0, column=0, sticky=E)
			
			restaurant_label = Label(roots, text='Restaurant-URL ')
			restaurant_label.grid(row=1, column=0,
			sticky=W)
			
			restaurant_entry = Entry(roots, width=150)
			
			restaurant_entry.grid(row=1, column=1)
			
			
			def check_url():
			base_restaurant_url_= "https://www.tripadvisor.de/Restaurant_Review"
			url_to_check = restaurant_entry.get()
			if  (not validators.url(url_to_check)) or \
			(base_restaurant_url_ not in url_to_check):
			
			messagebox.showwarning("Warning", "This seems not to be valid Tripadvisor restaurant URL")
			
			else:
			messagebox.showinfo("Vaildation successful", "Url seems to be valid")
			
			
			check_url_Button= Button(roots, text='Validate Url', command=check_url)
			
			def go_scrape():
			self.get_single_data(restaurant_entry.get())
			
			scrape_button = Button(roots, text='Go Scrape', command=go_scrape)
			
			check_url_Button.grid(columnspan=3, sticky=W)
			scrape_button.grid(columnspan=5, sticky=W)
			
			#adding Scraped Data as a Logfile displaying in a textbox
			self.txt = Text(roots, width=24, height=10)
			self.txt.grid(row=4, column=1)
			
			
			roots.mainloop()
			
			
			
			if __name__ == "__main__":
			ws = Web_scraping()
			ws.start_GUI()
			#ws.parse_to_tinydb()
			
			#url = "https://www.tripadvisor.de/Restaurant_Review-g946452-d8757235-Reviews-The_Forge_Tea_Room-Hutton_le_Hole_North_York_Moors_National_Park_North_Yorkshire_.html"
			
			
			     
			\end{lstlisting}

		
		
		
		
		
		
		
	\chapter{Blubber}\label{bluber}	
	
		Programmiersprache: C	
		usahdfiheaöoi
				
			

		

		
			
		
	
	
	
%						    	Hinzufügen relevanter Verzeichnisse

	\chapter*{Abkürzungsverzeichnis}
	\vspace{1.0cm}
	\begin{acronym}[SEPSEP]
	  	\acro{SD}[SD]{Secure Digital Memory Card}
	  	\acro{pc}[PC]{Personal Computer}
	  	%\acro{USB}[USB]{Universal Serial Bus}
	\end{acronym}
	\addcontentsline{toc}{chapter}{Abkürzungsverzeichnis}



\end{document}
%The End
%Johannes Knippel